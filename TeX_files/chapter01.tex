\chapter{Competitive programming}
\section{Introduction}
The core in ``competitive programming" is `` Given well-known Computer-Science problems, solve them quickly as possible!".
\newline
The term ``well-known CS problems" implies that in-competitive programming, we are dealing with solved problems and not research problems. so their must be at least one person solved this problem before. the idea behind this is to produce working code that can solve these problems with efficiency and within the time limit, so speed is critical and speed is already a very natural human behavior.
\newline
\bf Abridged Problem Description:
\newline
\normalfont 
Let (x, y) be the coordinates of a student's house on a 2D plane. There are 2N students
and we want to pair them into N groups. Let d i be the distance between the houses
of 2 students in group i. Form N groups such that cost = N
i=1 d i is minimized.
Output the minimum cost. Constraints: 1 ≤ N ≤ 8 and 0 ≤ x, y ≤ 1000.
\newline
Sample input:
N = 2; Coordinates of the 2N = 4 houses are {1, 1}, {8, 6}, {6, 8}, and {1, 3}.
\newline
Sample output:
cost = 4.83.
\newline
Can you solve this problem?\newline
If so, how many minutes would you likely require to complete the working code?\newline
